\documentclass{article}
\usepackage{graphicx}
\graphicspath{{images/}}

\title{Got It Diabetes Management:\\Mid-Point Design Document}
\date{2015-10-10}
\author{by a Coursera student}

\begin{document}
    \pagenumbering{gobble}
    \maketitle
    \newpage
    \pagenumbering{arabic}

\section{User-Facing Components}

    In this section of the document you will see the app from a user's perspective, that is the screens and major features the user can use. The technical and implementation details will be described in the next section.

\newpage

    \subsection{Authentication}

    A newly installed \emph{Diabetes Management} app should display the authentication screen to the user. Depending on whether the user has an existing account he/she needs to either log in or sign up (see the figure \ref{fig:screen_auth}).

    \begin{figure}[h]
        \centering
        \includegraphics[width=\textwidth,height=\textheight,keepaspectratio]{auth.png}
        \caption{authentication screens}
        \label{fig:screen_auth}
    \end{figure}

    \begin{description}
        \item[Log In] \hfill \\
            The user needs to enter their user name (or e-mail) and password. After hitting on the \emph{Log In} button the app will display the \emph{Check-In} screen (see the figure \ref{fig:screen_checkin}).
        \item[Sign Up] \hfill \\
            The registration process includes several steps, each of which gathers various user's information. Depending on the user type (\emph{Teen} or \emph{Follower}), the information gathered might differ. The \emph{Teen} account will at least include a first name, a last name, a date of birth, and a medical record number. A \emph{Followers} account will probably only include a first and a last name.
            \footnote{Functional Description and App Requirement \#1: The \emph{Teen} is the primary user of the mobile app and is represented in the app by a unit of data containing the core set of identifying information about a diabetic adolescent, including (but not necessarily limited to) a first name, a last name, a date of birth, and a medical record number.}
            Both account types will require an e-mail address.
        \item[Log Out] \hfill \\
            As you can see in the figure \ref{fig:screen_main}, the user can also log out if, for example, another person needs to use the app under their own account.
            \footnote{Basic Project Requirement \#1: App supports multiple users via individual user accounts}
    \end{description}

\newpage

    \subsection{Main Menu}

    To give the users ability to easily navigate between all the screens, the application uses a navigation drawer, which is called \emph{Main Menu}.
    
    Some menu items will indicated additional informations, for example, the number of unseen feedback, unread messages etc.

    \begin{figure}[h]
        \centering
        \includegraphics[width=0.5\textwidth,height=\textheight,keepaspectratio]{main.png}
        \caption{main menu}
        \label{fig:screen_main}
    \end{figure}


    This menu and the screens, which can be navigated to from the \emph{Main Menu}, are only available to authenticated users.
    \footnote{Basic Project Requirement \#2: App contains at least one user facing function available only to authenticated users}

\newpage

    \subsection{Check-In}

    This is the main screen for gathering a teen's diabetes related information. It might be open when the user clicks on a check-in notification or be navigated to from the \emph{Main Menu}. The figure \ref{fig:screen_checkin} shows what information will probably be required from a \emph{Teen} user.
    \footnote{Functional Description and App Requirement \#3: Once the \emph{Teen} acknowledges a Reminder, the app opens for a Check-In.  A Check-In includes data associated with that \emph{Teen}, a date, a time, and the user's responses to a set of Questions at that date and time.}

    \begin{figure}[h]
        \centering
        \includegraphics[width=\textwidth,height=\textheight,keepaspectratio]{checkin.png}
        \caption{check-in screen}
        \label{fig:screen_checkin}
    \end{figure}

    This screen is only available for the \emph{Teen} users. The \emph{Follower} users cannot preform check-in's.
    \footnote{Functional Description and App Requirement \#5: The app includes a \emph{Follower} role that is a different type of user (e.g., a parent, clinician, friend, etc.) who does not the ability to perform Check-Ins, but who can receive Check-In data shared from one or more \emph{Teens}. Also, the app allows a \emph{Teen} to be a \emph{Follower} for other \emph{Teens}.}

\newpage

    \subsection{Feedback}

    Both \emph{Teen} and \emph{Follower} accounts can receive feedback from the users they follow. This information will be displayed on the \emph{Feedback} screen (see the figure \ref{fig:screen_feedback}).
    
    Either the \emph{Followers} screen (figure \ref{fig:screen_followers}) or the \emph{Main Menu} (in case of a \emph{Teen} account to see their own feedback data) can be used to navigate to the \emph{Feedback} screen. 

    \begin{figure}[h]
        \centering
        \includegraphics[width=0.5\textwidth,height=\textheight,keepaspectratio]{feedback.png}
        \caption{feedback screen}
        \label{fig:screen_feedback}
    \end{figure}
    
    Whenever a \emph{Teen} submits check-in data, his/her followers will receive feedback via system notifications and/or notifications in the \emph{Main Menu}. 
    \footnote{Functional Description and App Requirement \#4: A \emph{Teen} is able to monitor their feedback data that is updated at some appropriate interval (e.g., when a Check-In is completed, daily, weekly, or when requested by \emph{Followers}). The Feedback data can be viewed graphically on the mobile device.}

    The amount of information will depend on the permissions the corresponding \emph{Teen} granted to their \emph{Follower(s)} (see the figures \ref{fig:screen_followers} and \ref{fig:screen_follow_req}).

\newpage

    \subsection{Followers}

    Each \emph{Teen} account user can have followers. These are users that receive feedback from the \emph{Teen} and can send him/her messages.

    \begin{figure}[h]
        \centering
        \includegraphics[width=\textwidth,height=\textheight,keepaspectratio]{followers.png}
        \caption{follower and settings screens}
        \label{fig:screen_followers}
    \end{figure}

    As you can see in the figure \ref{fig:screen_followers}, a \emph{Teen} can also change the amount of data he/she wants to share with a certain \emph{Follower}.
    \footnote{Functional Description and App Requirement \#6: The app allows a \emph{Teen} to choose what part(s) of their data to share with one or more \emph{Followers}.}

\newpage

In case if the \emph{Teen} has a new follower request, he/she can click on the corresponding item in the list (figure \ref{fig:screen_followers}) to open the \emph{New Follow Request} screen. The \emph{Teen} can  either accept or reject the request (see the figure \ref{fig:screen_follow_req}).
    
    If the \emph{Teen} wants to invite a \emph{Follower}, the \textbf{+} button can be used to open the corresponding screen, where the \emph{Teen} needs to enter the e-mail the person has used during the registration.
    \footnote{Functional Description and App Requirement \#7: The app only allows \emph{Teen} data to be disseminated to authorized/authenticated \emph{Followers} and accessed over HTTPS to enhance privacy and security of the data (the HTTPS part will be described in the implementation details)}
    The person, whom the invite has been sent to, will receive a notification where he/she can accept/reject the invitation.

    \begin{figure}[h]
        \centering
        \includegraphics[width=\textwidth,height=\textheight,keepaspectratio]{follow_req.png}
        \caption{follow request and invitation screens}
        \label{fig:screen_follow_req}
    \end{figure}

\newpage

    \subsection{Following}

    To see the list of the users you currently follow the \emph{Following} screen can be used. The user can access feedback of the users he/she follows or start to follow another one by clicking on the \textbf{+} button (see the figure \ref{fig:screen_following}).

    \begin{figure}[h]
        \centering
        \includegraphics[width=\textwidth,height=\textheight,keepaspectratio]{following.png}
        \caption{following screen}
        \label{fig:screen_following}
    \end{figure}

\newpage

    \subsection{Reminders}

    Each \emph{Teen} will receive check-in reminders in the form of alarms at least three times a day. Once a reminder has been received a \emph{Teen} can tap on the notification and the \emph{Check-In} screen will be open to perform the check-in process.
    Reminder can be added, edited or tuned on/off  using the \emph{Reminders} screen (see the figure \ref{fig:screen_reminders}). At least three reminders will always be on.
    \footnote{Functional Description and App Requirement \#2: The \emph{Teen} receives a Reminder in the form of alarms or notifications at patient-adjustable times at least three times per day.}

    \begin{figure}[h]
        \centering
        \includegraphics[width=0.5\textwidth,height=\textheight,keepaspectratio]{reminders.png}
        \caption{reminders screen}
        \label{fig:screen_reminders}
    \end{figure}

\newpage

    \subsection{Optional Features}

    The following features will be implemented only if the major capstone project requirements are met and there is development time left.

    \begin{itemize}
        \item Messages

            \emph{Teen} account will be able to communicate with their \emph{Followers} and vice versa by using the \emph{Messages} screen. A \emph{Teen} or their \emph{Follower(s)} will be able to communicate in the form of a dialog (see the figure \ref{fig:screen_messages}).


            \begin{figure}[h]
                \centering
                \includegraphics[width=\textwidth,height=\textheight,keepaspectratio]{messages.png}
                \caption{messages and dialog screens}
                \label{fig:screen_messages}
            \end{figure}

        \item User Settings (the screen is not present)
    \end{itemize}
            This screen will allow to alter different user information and settings. For example, a user could change the password or turn some notifications off.

\newpage

\section{Implementation Details}

\end{document}
